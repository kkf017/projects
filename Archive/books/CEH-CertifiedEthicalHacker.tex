\documentclass[a4paper]{article}
%\usepackage[utf8]{inputenc}
\usepackage[T1]{fontenc}

\usepackage{graphicx}
\usepackage{hyperref}


\title{%
  CEH \\
  \large Certified Ethical Hacker \\
    (Study Guide)}
    
 \author{}


\begin{document}

\maketitle

\section*{Assessment test (xxxvii)}
A \textbf{replay attack} occurs when a cybercriminal eavesdrops on a secure network communication, intercepts it, and then fraudulently delays or resends it to misdirect the receiver into doing what the hacker wants. The added danger of replay attacks is that a hacker doesn't even need advanced skills to decrypt a message after capturing it from the network. The attack could be successful simply by resending the whole thing. \\


Une \textbf{attaque par smurf} est une forme d’attaque par déni de service distribué (DDoS) qui se produit au niveau de la couche réseau. Les attaques par smurfing portent le nom du malware DDoS.Smurf, qui permet aux hackers de les exécuter. Plus largement, les attaques sont nommées d’après les personnages du dessin animé The Smurfs en raison de leur capacité à détruire des ennemis plus grands en travaillant ensemble.

Les attaques par smurf DDoS sont similaires en style aux inondations de ping, qui sont une forme d’attaque par déni de service (DoS). Un hacker surcharge les ordinateurs avec des demandes d’ écho ICMP ( Internet Control Message Protocol), également appelées pings. L’ICMP détermine si les données atteignent la destination prévue au bon moment et surveille la manière dont un réseau transmet les données. Une attaque par smurf envoie également des pings ICMP, mais elle est potentiellement plus dangereuse, car elle peut exploiter les vulnérabilités du protocole Internet (IP) et de l’ICMP.



\end{document}
